\begin{center}
Abstract of the Dissertation \\ \vspace{1em}
\textbf{Tool-assisted induction of subregular languages and mappings} \\ \vspace{1em}
by \\ \vspace{1em}
\textbf{Al\"ena Aks\"enova} \\ \vspace{1em}
\textbf{Doctor of Philosophy} \\ \vspace{1em}
in \\ \vspace{1em}
\textbf{Linguistics} \\ \vspace{1em}
Stony Brook University \\ \vspace{1em}
\textbf{2020} \\ \vspace{3em}
\end{center}



The last decade was very fruitful in the field of subregular research.
New classes of subregular languages and mappings were uncovered for modeling natural language phenomena, and new learning algorithms were developed for these classes.
The subregular approach has been successfully applied to phonotactics \citep{Heinz10ldp}, rewrite processes in phonology and morphology \citep{Chandlee2014}, and even syntactic constraints over tree structures \citep{Graf18CLS}.
However, the rapid pace of the theoretical research has not been matched when it comes to engineering considerations.
Many of the proposed learning algorithms have not been implemented yet, and as a result, their performance on concrete data sets is not known.

In my dissertation, I implement and experiment with some of the learners available for subregular languages and mappings.
I test these learners on data that is modeled after linguistic phenomena such as word-final devoicing and various types of harmony systems.
The code for these evaluations is available as part of my Python package \emph{SigmaPie} \href{https://pypi.org/project/SigmaPie/}{\faCube} \citep{sigmapie}.

The findings of my thesis allow linguists and formal language theorists to assess possible applications of subregular techniques and approaches, in particular typology, cognitive science, and natural language processing.

\clearpage
\thispagestyle{plain}
\par\vspace*{.35\textheight}{\flushright \emph{There is a single light of science, and \\ to brighten it anywhere is to brighten it everywhere.}\par
--- Isaac Asimov
\par}